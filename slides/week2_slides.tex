\documentclass[aspectratio=169]{beamer}
\usetheme{Madrid}
\usecolortheme{default}

\usepackage{graphicx}
\usepackage{booktabs}
\usepackage{amsmath}

\title{Week 2: Statistical Analysis}
\subtitle{Heart Failure Survival Analysis}
\author{MDST Project}
\date{Winter 2026}

\begin{document}

\begin{frame}
\titlepage
\end{frame}

\begin{frame}{Outline}
\tableofcontents
\end{frame}

\section{Correlation Analysis}

\begin{frame}{Pearson Correlation}
\begin{itemize}
    \item Measures \textbf{linear relationship} between two variables
    \item Range: $-1$ to $+1$
    \begin{itemize}
        \item $+1$: Perfect positive correlation
        \item $0$: No linear correlation
        \item $-1$: Perfect negative correlation
    \end{itemize}
    \item Formula: $r = \frac{\sum(x_i - \bar{x})(y_i - \bar{y})}{\sqrt{\sum(x_i - \bar{x})^2 \sum(y_i - \bar{y})^2}}$
\end{itemize}
\end{frame}

\begin{frame}{Feature Correlations with DEATH\_EVENT}
\begin{columns}
\column{0.5\textwidth}
\textbf{Positive Correlations:}
\begin{itemize}
    \item serum\_creatinine: 0.29
    \item age: 0.25
\end{itemize}

\vspace{0.5cm}
\textbf{Negative Correlations:}
\begin{itemize}
    \item time: -0.53
    \item ejection\_fraction: -0.27
    \item serum\_sodium: -0.20
\end{itemize}

\column{0.5\textwidth}
\textbf{Key Insight:}\\
Follow-up time has the strongest correlation with death event, but this is expected (patients who die have shorter follow-up).
\end{columns}
\end{frame}

\section{Statistical Tests}

\begin{frame}{T-Test}
\begin{itemize}
    \item Compares \textbf{means} between two groups
    \item Assumes: Normal distribution, equal variances
    \item Welch's t-test: Does not assume equal variances
    \item Null hypothesis: No difference between group means
\end{itemize}

\vspace{0.5cm}
\textbf{When to use:}
\begin{itemize}
    \item Continuous data
    \item Comparing two groups (survived vs. died)
\end{itemize}
\end{frame}

\begin{frame}{Mann-Whitney U Test}
\begin{itemize}
    \item \textbf{Non-parametric} alternative to t-test
    \item Does NOT assume normal distribution
    \item Compares \textbf{ranks} instead of means
    \item More robust to outliers
\end{itemize}

\vspace{0.5cm}
\textbf{When to use:}
\begin{itemize}
    \item Data is not normally distributed
    \item Ordinal data or skewed distributions
\end{itemize}
\end{frame}

\begin{frame}{Significant Features (p < 0.05)}
\begin{table}
\centering
\begin{tabular}{lcc}
\toprule
\textbf{Feature} & \textbf{T-test p-value} & \textbf{Mann-Whitney p-value} \\
\midrule
time & $2.3 \times 10^{-22}$ & $6.9 \times 10^{-21}$ \\
ejection\_fraction & $9.6 \times 10^{-6}$ & $7.4 \times 10^{-7}$ \\
age & $4.7 \times 10^{-5}$ & $1.7 \times 10^{-4}$ \\
serum\_creatinine & $6.4 \times 10^{-5}$ & $1.6 \times 10^{-10}$ \\
serum\_sodium & $1.9 \times 10^{-3}$ & $2.9 \times 10^{-4}$ \\
\bottomrule
\end{tabular}
\end{table}
\end{frame}

\section{Multiple Testing Correction}

\begin{frame}{The Problem with Multiple Testing}
\begin{itemize}
    \item Testing 12 features at $\alpha = 0.05$
    \item Each test has 5\% chance of false positive
    \item Expected false positives: $12 \times 0.05 = 0.6$
    \item \textbf{Family-wise error rate} increases with more tests
\end{itemize}

\vspace{0.5cm}
\textbf{Solution:} Adjust p-values to control false discovery rate (FDR)
\end{frame}

\begin{frame}{Benjamini-Hochberg (FDR) Correction}
\begin{itemize}
    \item Controls the \textbf{False Discovery Rate}
    \item FDR = Expected proportion of false positives among rejected hypotheses
    \item Less conservative than Bonferroni correction
    \item Procedure:
    \begin{enumerate}
        \item Rank p-values from smallest to largest
        \item Adjust: $p_{adj} = p \times \frac{n}{rank}$
        \item Compare adjusted p-values to $\alpha$
    \end{enumerate}
\end{itemize}
\end{frame}

\begin{frame}{After FDR Correction}
\textbf{Still significant (FDR < 0.05):}
\begin{itemize}
    \item time
    \item ejection\_fraction
    \item age
    \item serum\_creatinine
    \item serum\_sodium
\end{itemize}

\vspace{0.5cm}
\textbf{Not significant after correction:}
\begin{itemize}
    \item high\_blood\_pressure, anaemia, diabetes, platelets, sex, smoking, creatinine\_phosphokinase
\end{itemize}
\end{frame}

\section{Summary}

\begin{frame}{Key Takeaways}
\begin{enumerate}
    \item \textbf{Correlation} measures linear relationships
    \item \textbf{T-test} compares means (assumes normality)
    \item \textbf{Mann-Whitney U} is non-parametric (no normality assumption)
    \item \textbf{Multiple testing correction} is essential when testing many hypotheses
    \item \textbf{5 features} are significantly different between survival groups
\end{enumerate}
\end{frame}

\begin{frame}{Next Week: Unsupervised Learning}
\begin{itemize}
    \item \textbf{PCA} (Principal Component Analysis)
    \begin{itemize}
        \item Dimensionality reduction
        \item Visualizing high-dimensional data
    \end{itemize}
    \item \textbf{Clustering}
    \begin{itemize}
        \item Finding natural groupings in data
        \item K-means, hierarchical clustering
    \end{itemize}
\end{itemize}
\end{frame}

\begin{frame}{Exercises}
\begin{enumerate}
    \item Write a function to return features with significant p-values given a threshold
    \item Implement Mann-Whitney U test for all features
    \item Interpret the correlation heatmap
\end{enumerate}

\vspace{0.5cm}
\textbf{Resources:}
\begin{itemize}
    \item Scipy Stats: \url{https://docs.scipy.org/doc/scipy/reference/stats.html}
    \item Statsmodels: \url{https://www.statsmodels.org/}
\end{itemize}
\end{frame}

\end{document}
